\documentclass{article}
\usepackage[a4paper]{geometry}
\usepackage{graphicx}
\usepackage[utf8]{inputenc}
\usepackage{subcaption}
\usepackage{placeins}
\usepackage{wrapfig}
\usepackage{float}
\usepackage{hyperref}
\usepackage{minted}

\graphicspath{{./images/}}

\title{Aplicație pentru managementul unei biblioteci}
\author{
    Mițca Dumitru\\
    Grupa 1311A\\
    Coordonator: Mironeanu Cătălin\\
    \emph{Facultatea de Automatică și Calculatoare}
}
\date{2023}

\begin{document}
    \maketitle
    \hypersetup{linkbordercolor=1 1 1}
    \renewcommand*\contentsname{Cuprins}
    \tableofcontents
    \hypersetup{linkbordercolor=1 0 0}

    \newpage

    \section{Descrierea proiectului}

    \subsection{Scopul aplicației}

    Aplicația a fost concepută pentru management-ul unei biblioteci, cu funcționalități utile și pentru
    oamenii care doar împrumută cărți, cât și pentru bibliotecari.

    Astfel, pentru utilizatorii normali, sunt oferite următoarele funcționalități:
    \begin{itemize}
        \item abilitatea de a vedea toate cărțile din bibliotecă cu detalii despre carte și autor
        \item abilitatea de a împrumuta o carte, și de a vedea ușor data la care cartea trebuie înapoiată bibliotecii
        \item abilitatea de a-și menține progresul citirii cărții in aplicație prin numărul de capitole citite
    \end{itemize}
    Ca o posibilitate de extindere în viitor a aplicației ar fi adăugare de funcționalități sociale, care ar permite
    utilizatorilor să vadă ce cărți citesc prietenii lor, sau ca o carte să treacă direct de la un utilizator
    la altul.

    Pentru bibliotecari, următoarele funcționalități sunt oferite:
    \begin{itemize}
        \item
    \end{itemize}

    \subsection{Arhitectura proiectului}

    Aplicația este separată în două programe: un server web și o aplicație desktop care comunică
    prin un API REST. Aplicația desktop nu are acces la baza de date, lucru rezervat exclusiv
    serverului.

    \section{Tehnologii folosite}

    Clientul și serverul sunt amândouă scrise în limbajul de programare Rust\footnote{\url{https://www.rust-lang.org/}},
    lucru care permite împărtășirea definițiilor structurilor folosite în API prin plasarea lor într-o librărie.
    Aceste structuri sunt transmise prin rețea serializându-le în JSON.

    \subsection{Pe back-end}

    Back-end-ul constă într-un server care folosește librăria \emph{axum} pentru expunerea unui API REST,
    librăria \emph{SQLx} pentru comunicația cu baza de date și \emph{SQLite} pentru baza de date în sine.

    \subsubsection*{\emph{axum}\footnote{\url{https://github.com/tokio-rs/axum}}}

    \subsubsection*{\emph{SQLx}\footnote{\url{https://github.com/launchbadge/sqlx}}}

    \subsubsection*{\emph{SQLite}\footnote{\url{https://www.sqlite.org/index.html}}}

    \subsection{Pe front-end}

    Front-end-ul constă într-o aplicație desktop care folosește \emph{GTK}\footnote{\url{https://www.gtk.org/}}
    și \emph{Adwaita}\footnote{\url{https://gnome.pages.gitlab.gnome.org/libadwaita/doc/}} pentru interfață și
    \emph{Soup}\footnote{\url{https://libsoup.org/libsoup-3.0/index.html}} pentru a realiza request-urile HTTP
    pentru comunicarea cu serverul.

    Am ales să folosesc GTK deoarece există suport foarte bun pentru utilizarea librăriei în Rust și pentru
    că împreuna cu \emph{Blueprints}\footnote{\url{https://jwestman.pages.gitlab.gnome.org/blueprint-compiler/}},
    am putut scrie definiția interfeței grafice într-un mod declarativ, Rust fiind delegat doar pentru logica
    aplicației.

    \emph{Soup} a fost ales deoarece oferă o integrare bună cu GTK, fiind o librărie care își are originile
    în ecosistemul GTK.

    \section{Structura bazei de date}

\end{document}
