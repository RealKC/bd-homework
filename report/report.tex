\documentclass{article}
\usepackage[a4paper]{geometry}
\usepackage{graphicx}
\usepackage[utf8]{inputenc}
\usepackage{subcaption}
\usepackage{placeins}
\usepackage{wrapfig}
\usepackage{float}
\usepackage{hyperref}
\usepackage{minted}

\graphicspath{{./images/}}

\title{Aplicație pentru managementul unei biblioteci}
\author{
    Mițca Dumitru\\
    Grupa 1311A\\
    Coordonator: Mironeanu Cătălin\\
    \emph{Facultatea de Automatică și Calculatoare}
}
\date{2023}

\begin{document}
    \maketitle
    \hypersetup{linkbordercolor=1 1 1}
    \renewcommand*\contentsname{Cuprins}
    \tableofcontents
    \hypersetup{linkbordercolor=1 0 0}

    \newpage

    \section{Descrierea proiectului}

    \subsection{Scopul aplicației}

    Aplicația a fost concepută pentru management-ul unei biblioteci, cu funcționalități utile și pentru
    oamenii care doar împrumută cărți, cât și pentru bibliotecari.

    Astfel, pentru utilizatorii normali, sunt oferite următoarele funcționalități:
    \begin{itemize}
        \item abilitatea de a vedea toate cărțile din bibliotecă cu detalii despre carte și autor
        \item abilitatea de a împrumuta o carte, și de a vedea ușor data la care cartea trebuie înapoiată bibliotecii
        \item abilitatea de a-și menține progresul citirii cărții in aplicație prin numărul de capitole citite
    \end{itemize}
    Ca o posibilitate de extindere în viitor a aplicației ar fi adăugare de funcționalități sociale, care ar permite
    utilizatorilor să vadă ce cărți citesc prietenii lor, sau ca o carte să treacă direct de la un utilizator
    la altul.

    Pentru bibliotecari, următoarele funcționalități sunt oferite:
    \begin{itemize}
        \item abilitatea de a vedea toate cărțile, a le edita și a le șterge
        \item abilitatea de a adăuga autori
        \item abilitatea de a termina și prelungi durata împrumuturilor cărților
        \item abilitatea de a șterge utilizatori
    \end{itemize}

    \subsection{Arhitectura proiectului}

    Aplicația se încadrează în categoria aplicațiilor CRUD.

    Aplicația este separată în trei module:
    \begin{itemize}
        \item o librărie comună (în folderul \texttt{bd-schema}) care conține definițiile structurilor
        transmise prin API
        \item un server web REST (în folderul \texttt{server})
        \item o aplicație desktop (în folderul \texttt{app}) care comunică cu server-ul; aplicația nu are access la baza de date
    \end{itemize}

    \section{Tehnologii folosite}

    Clientul și serverul sunt amândouă scrise în limbajul de programare Rust\footnote{\url{https://www.rust-lang.org/}},
    lucru care permite împărtășirea definițiilor structurilor folosite în API prin plasarea lor într-o librărie.
    Aceste structuri sunt transmise prin rețea serializându-le în JSON.

    \subsection{Pe back-end}

    Back-end-ul constă într-un server care folosește librăria \emph{axum} pentru expunerea unui API REST,
    librăria \emph{SQLx} pentru comunicația cu baza de date și \emph{SQLite} pentru baza de date în sine.

    \subsubsection*{\emph{axum}\footnote{\url{https://github.com/tokio-rs/axum}}}

    \emph{axum} este un framework pentru servere web ce permite descrierea rutelor expuse de server într-un mod
    declarativ. Acest framework este foarte simplu de utilizat și nu impune mult boilerplate, un exemplu de definire
    a rutelor este:
    \begin{minted}[linenos, breaklines]{rust}
let app = Router::new()
    .route("/books", get(books::books))
    .route("/borrow", post(books::borrow))
    .route("/borrowed-by/:user_id", post(books::borrowed_by))
    .nest("/auth", auth::router(pool.clone()))
    .fallback(fallback) // această funcție permite returnarea unui răspuns HTTP cu codul 404 automat pentru rute necunoscute
    .with_state(pool);
    \end{minted}

    \emph{axum} permite și definirea în prototipul funcțiilor formatul în care sunt serializate request-urile de la client
    și răspunsul de la server, ocupându-se automat de serializarea și deserializarea acestora, spre exemplu:
    \begin{minted}[linenos, breaklines]{rust}
async fn create_account(
    State(pool): State<SqlitePool>,
    Json(data): Json<CreateAccount>,
) -> Result<Json<LoginReply>, RouteError>;
    \end{minted}

    \subsubsection*{\emph{SQLx}\footnote{\url{https://github.com/launchbadge/sqlx}}}

    \emph{SQLx} este o librărie care permite conectarea la baze de date și scrierea de interogări SQL.
    Particularitatea acestei librăriei este integrarea cu funcționalitățile pentru asincronicitate în
    Rust\footnote{Totuși, datorită utilizării \emph{SQLite}, back-end-ul nu beneficiază total de acest lucru,
    dar dacă ar fi să schimb baza de date, ar reprezenta un avantaj.} și posibilitatea verificării în timpul
    compilării că interogările scrise de programator sunt în coconcordanță cu schema bazei de date definită
    în fișiere SQL de programator.

    \subsubsection*{\emph{SQLite}\footnote{\url{https://www.sqlite.org/index.html}}}

    \emph{SQLite} este o bază de date relațională, implementată ca o librărie care rulează în același proces ca
    aplicație ce dorește să o folosească. Deși are "lite" în nume, \emph{SQLite} este o implementare completă a
    SQL\footnote{Cu mențiunea că suportul pentru foreign key-uri trebuie activat per conexiune, lucru realizat de
    back-end.} și oferă tranzacției care respectă proprietățile ACID.

    Conexiunea către baza de date este realizată prin intermediul \emph{SQLx}, care pur și simplu apelează funcția
    din \emph{SQLite} pentru deschiderea unei baze de date.

    \subsection{Pe front-end}

    Front-end-ul constă într-o aplicație desktop care folosește \emph{GTK}\footnote{\url{https://www.gtk.org/}}
    și \emph{Adwaita}\footnote{\url{https://gnome.pages.gitlab.gnome.org/libadwaita/doc/}} pentru interfață și
    \emph{Soup}\footnote{\url{https://libsoup.org/libsoup-3.0/index.html}} pentru a realiza request-urile HTTP
    pentru comunicarea cu serverul.

    Am ales să folosesc GTK deoarece există suport foarte bun pentru utilizarea librăriei în Rust și pentru
    că împreuna cu \emph{Blueprints}\footnote{\url{https://jwestman.pages.gitlab.gnome.org/blueprint-compiler/}},
    am putut scrie definiția interfeței grafice într-un mod declarativ, Rust fiind delegat doar pentru logica
    aplicației.

    \emph{Soup} a fost ales deoarece oferă o integrare bună cu GTK, fiind o librărie care își are originile
    în ecosistemul GTK.

    \section{Structura bazei de date}

    \begin{figure}[H]
        \includegraphics[width=\linewidth]{ER-Diagram}
        \centering
        \caption{Diagrama ER}
    \end{figure}

    Mențiuni despre normalizarea datelor:
    \begin{itemize}
        \item prima formă normală --- toate tabele conțin coloane ce au valori atomice și nu conțin grupuri care se repetă
        \item a doua formă normală --- toate tabele cu chei primare, au chei primare formate dintr-o singură coloană și sunt
        în prima formă normală tabela \texttt{BorrowData} este și ea în a doua formă normala deoarece are o singură cheie
        candidat \texttt{borrow\_id} și este în prima formă normală
        \item a treia formă normală --- toate tabelele sunt în a doua formă normală și toate coloanele non-cheie sunt direct
        dependente de cheia candidat
        \item forma normală Boyce-Codd --- nu am luat-o în considerare la proiectare bazei de date și nu cred că a apărut accidental;
        nu consider acest lucru un dezavantaj
        \item a patra formă normală și a cincea formă normală --- probabil că nu, nu au fost luate în considerare la proiectarea
        bazei de date
        \item nu a fost necesară încălcarea deliberată a unei forme normale pentru performanță deoarece aplicația este mică
    \end{itemize}
    Constrângeri utilizate în baza de date:
    \begin{itemize}
        \item constrângeri de tip cheie primară --- majoritatea tabelelor au chei primare, generate la inserare prin
        funcționalitatea \texttt{AUTOINCREMENT} a \emph{SQLite} și unicitatea lor este asigurată prin utilizarea
        constrăngerii \texttt{PRIMARY KEY} aplicată coloanelor relevante la crearea tabelelor
        \item constrângeri de tip cheie străină --- datorită existența relațiilor între tabele, de exemplu: utilizator
        la cărți împrumutate sau carte la autor; am folosit constrângerea \texttt{FOREIGN KEY} pentru a asigura coerența
        datelor, un exemplu de situație prevenită astfel este ștergerea unui autor când încă există cărți scrise de el
        în baza de date
        \item constrăngeri de tipul \texttt{NOT NULL} --- am folosit astfel de constrângeri pentru că pentru majoritatea
        coloanelor din baza de date, absența valorilor nu are sens
        \item o constrângere de tipul \texttt{UNIQUE} --- în tabela \texttt{BorrowData}, pentru a asigura că nu există mai
        multe înregistrări cu date auxiliare pentru același împrumut
    \end{itemize}

    \section{Capturi de ecran din aplicație împreună cu fragmente din codul care le face funcționale}

    \begin{figure}[H]
        \includegraphics[width=\linewidth]{Borrowed-Books}
        \centering
        \caption{Pagina care îi permite unui utilizator să vadă ce cărți are împrumutate, cu detaliile aferente}
    \end{figure}
    \begin{minted}[linenos, breaklines]{rust}
// În o librărie folosită de aplicația desktop și server
#[derive(Serialize, Deserialize)] // Permite (de)serializarea structurii
pub struct BorrowedBook {
    pub borrow_id: Integer,
    pub book_id: Integer,
    pub valid_until: Integer,
    pub chapters_read: Integer,
}
// În aplicația desktop
// Această funcție este apelată inițial la afișarea paginii, și apoi când utilizatorul apasă butonul de "Reîmprospătare"
async fn refresh_borrowed_books(&self) {
    // Comunicația cu serverul
    let books = self
        .soup_session()
        .post::<BorrowedByReply>("", &format!("/borrowed-by/{}", self.cookie().user_id()))
        .await;

    match books {
        Ok(books) => {
            // În cazul de success, actualizez datele, GTK se ocupă automat de actualizarea widgeturilor
            self.borrowed_books.remove_all();
            let books = books
                .into_iter()
                .map(BoxedAnyObject::new)
                .collect::<Vec<_>>();
            self.borrowed_books.extend_from_slice(&books);
        }
        Err(err) => {
            // .. Tratarea erorilor ..
        }
    }
}
// Pe server
pub async fn borrowed_by(
    Path(user_id): Path<i64>,
    State(pool): State<SqlitePool>,
) -> Result<Json<BorrowedByReply>, RouteError> {
    let records = sqlx::query!(
        r#"
SELECT d.borrow_id, b.book_id, d.valid_until, d.chapters_read
FROM Borrows b JOIN BorrowData d ON b.borrow_id = d.borrow_id
WHERE b.user_id = ?
    "#,
        user_id
    )
    .fetch_all(&pool)
    .await
    .unwrap();

    let data = records
        .into_iter()
        .map(|record| BorrowedBook {
            borrow_id: record.borrow_id,
            book_id: record.book_id,
            valid_until: record.valid_until,
            chapters_read: record.chapters_read,
        })
        .collect();

    Ok(Json(data))
}
    \end{minted}

    \begin{figure}[H]
        \includegraphics[width=0.5\linewidth]{New-Book}
        \centering
        \caption{Dialogul ce permite editarea sau adăugarea unei cărți noi, în acest caz utilizatorul a deschis dialogul pentru adăugarea unei cărți}
    \end{figure}
    \begin{minted}[linenos, breaklines]{rust}
// În o librărie folosită de aplicația desktop și server
#[derive(Serialize, Deserialize)]
pub struct ChangeBookDetailsRequest {
    pub book_id: Option<Integer>,
    pub title: Text,
    pub author_id: Integer,
    pub publish_date: Integer,
    pub publisher: Text,
    pub count: Integer,
    pub synopsis: Text,
    pub cookie: session::Cookie,
}

// În aplicația desktop, o funcție callback atașată butonului "Salvează schimbările"
async fn on_save_changes_clicked(&self, widget: gtk::Button) {
    let request = ChangeAuthorDetailsRequest {
        .. // structura din librărie este construită luând și validând valorile din widget-uri
    };

    // comunicația efectivă cu serverul
    let result = librarian_view
        .soup_session()
        .post::<()>(request, "/change-author-details")
        .await;

    // tratarea erorilor
}

// Pe server
pub async fn change_book_details(
    State(pool): State<SqlitePool>,
    Json(request): Json<ChangeBookDetailsRequest>,
) -> Result<(), RouteError> {
    verify_user_is_librarian(&pool, request.cookie).await?;
    if let Some(book_id) = request.book_id {
        sqlx::query!("
UPDATE Books SET title = ?, author_id = ?, publish_date = ?, publisher = ?,
                 count = ?, synopsis = ?
WHERE book_id = ?
        ",
            request.title, request.author_id, request.publish_date,
            request.publisher, request.count, request.synopsis, book_id
        )
        .execute(&pool)
        .await
        .http_internal_error("Failed to update book")?;
    } else {
        sqlx::query!(
            "
INSERT INTO Books(title, author_id, publish_date, publisher, count, synopsis, language)
VALUES (?, ?, ?, ?, ?, ?, 'ro')
        ",
            request.title, request.author_id, request.publish_date,
            request.publisher, request.count, request.synopsis,
        )
        .execute(&pool)
        .await
        .http_internal_error("Failed to insert book")?;
    }
    Ok(())
}
    \end{minted}



\end{document}
